\chapter{Introduction}		% chapter 1
\label{introchap}		% for reference (\ref{introchap})

This sample document illustrates how to use the
{\tt niuthesis} class.
We always put a bit of text in between the heading commands to see how it goes.
\emph{We always put a bit of text in between the heading commands to see how it goes.}

\section{This Is a Section (Level 2)}

First we want to see how the sectioning commands work on different
levels of sectioning.
We always put a bit of text in between the heading commands to see how it goes.
\textbf{We always put a bit of text in between the heading commands to see how it goes.}

\subsection{So Here We Have a Subsection (Level 3)}

We always put a bit of text in between the heading commands to see how it goes.
We always put a bit of text in between the heading commands to see how it goes.
We always put a bit of text in between the heading commands to see how it goes.

\subsubsection{And Here a \LaTeX\ Subsubsection  (Level 4)}

We always put a bit of text in between the heading commands to see how it goes.
We always put a bit of text in between the heading commands to see how it goes.
We always put a bit of text in between the heading commands to see how it goes.

\paragraph{And Here Yet Lower Sectioning, a Paragraph (Level 5)}

We always put a bit of text in between the heading commands to see how it goes.
We always put a bit of text in between the heading commands to see how it goes.
We always put a bit of text in between the heading commands to see how it goes.

%% subparagraphs work fine. But they should not appear in an
%% official NIU thesis.
%% \subparagraph{Does anyone ever need subparagraphs?  (Level 6)} 

%% Well we provide subparagraphs anyway because they are part of \LaTeX.
%% We always put a bit of text in between the heading commands to see how it goes.
%% We always put a bit of text in between the heading commands to see how it goes.

\section{This Is a Section (Level 2)}

First we want to see how the sectioning commands work on different
levels of sectioning.
We always put a bit of text in between the heading commands to see how it goes.
We always put a bit of text in between the heading commands to see how it goes.

\section{This Is a Section (Level 2)}

First we want to see how the sectioning commands work on different
levels of sectioning.
We always put a bit of text in between the heading commands to see how it goes.
We always put a bit of text in between the heading commands to see how it goes.

\section{This Is a Long Section Title That Needs to Be Broken Over
 Two Lines -- or May Be Three? We Will See \ensuremath{\ldots}}

Some requirements of the Graduate School are written
into that file; page size, line spacing, appropriate
placement of captions for tables and figures, etc.
Other tasks of conforming to the requirements are
left to other existing \LaTeX{} packages.

\subsection{Question:  What Are the Issues in Studying This Subject?}

A major goal in studying solid fuel rocket motors is to create a model
of the dynamics of a motor chamber.  This involves two major goals:
the combustion zone and the acoustic zone.

The combustion zone consists of the thin layer above the solid fuel
where the gasification of the fuel takes places. The zone is very
reactive and highly turbulent. The acoustic-vortical zone is the
volume of gas above the combustion zone. Within this zone, the gas
is non-reactive and contains acoustic waves and vorticity. The work
presented here\footnote{Footnotes are handled neatly by \LaTeX.} is
an extension of Lao \cite{lao:thesis} and Lao et~al.\
\cite{lao:paper}. The driving frequency is on the order of the
inverse of the axial acoustic time scale, $t_A'= L'/C_0'$, where
$L'$ is the length of the cylinder and $C_0'$ is the reference speed
of sound.\footnote{Remember the traditional method of calculating
 the distance of lightning? See the flash, count seconds until you
 hear the thunder, divide by five, that's the number of miles. That
 assumes $C_0=\frac{1 mi.}{5 s}$.} Radial and azimuthal velocities
are found to vanish exponentially fast in the downstream direction,
as suggested by Table \ref{powertable}.

\begin{table}[htb]
\caption[Example of a table with its own footnotes]{\label{powertable}
	Here is an example of a table with its own footnotes.
	Don't use the $\backslash${\tt footnote} macro if you
	don't want the footnotes at the bottom of the page.
	Also, note that in a thesis the caption goes
	\emph{above} a table, unlike figures.
	}
\begin{center}
\begin{tabular}{||l|c|c|c|c||} \hline
	& $S$ & $P$ &   $Q^{\ast}$  & $D^{\dagger}$ \\	% footnote symbols!
	wave form & (kVA) & (kW) & (kVAr) & (kVAd) \\  \hline \hline
	& 25.87 & 25.83 & 1.3 & $\approx 0$ \\ \hline
	& 25.48 & 25.00 & -2.82 & 4.03 \\ \hline
	& 25.11 & 18.02 & -9.75 & 14.52 \\ \hline
	Table \ref{tbl:sample2}  & 24.98 & 22.26 & 9.19 & 6.64 \\ \hline
	Fig.  \ref{tbl:sidewaysT}  & 23.48 & 15.00 & 6.59 & 16.82 \\ \hline
	Fig.  \ref{fig:pyramid}  & 24.64 & 22.81 & -0.44 & 9.3 \\ \hline
	Fig.  \ref{fig:sidewaysF}  & 23.03 & 18.01 & 3.36 & 13.95 \\ \hline
	\end{tabular}
   \\ \rule{0mm}{5mm}
   ${}^\ast$kVAr means reactive power.		% footnote symbol
\\ ${}^\dagger$kVAd means distortion power.	% footnote symbol
\end{center}
\end{table}

These results provide an analytical explanation of those
found from computational analysis by Fabnis
et~al.\ \cite{fabnis}.  The non-axisymmetric flow near the
endwall contains cross-sectional velocity patterns that
include flow across the cylinder axis.  A viscous boundary
layer adjacent to the sidewall and near the endwall is
studied to find the transition between the transient core
flow and the no-slip condition on the sidewall.
It is found, as in Lao et~al.\ \cite{lao:paper}, that the
azimuthal component of the vorticity is proportional to
the inverse of the Mach number.  In addition, the axial
component of the vorticity driven by the non-axisymmetric
boundary condition at the endwall is also found to be
proportional to the the inverse of the Mach number.

%%%%%%%%%%%%%%%%%%%%%%%%%%%%%%%%%%%%%%%%%%%%%%%%%%%%%%%%%%%%%%%%%%

%%% Local Variables: 
%%% TeX-master: "mythesis"
%%% End: 
